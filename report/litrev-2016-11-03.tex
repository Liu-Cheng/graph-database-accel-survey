\section{GraphMat}
On top of the vertex programming model, this work \cite{sundaram2015graphmat} focuses on transforming graph
processing to sparse matrix computing. With the well-studied matrix computing on
HPC, GraphMat achieves very good performance and scales on multi-core systems.

With this framework, it is possible to develop graph processing accelerator over
spare matrix accelerator which is also well-studied. 

Although a few classical graph algorithms including BFS and SSSP can be
transformed to spare matrix computing, there is no guarantee for general graph
problems. 

\section{In-memory Graph Database for Web-scale Data}
In this work \cite{Castellana2015in-memory}, Graph Database Engine for Multithreaded System (GEMS) is developed
for implementing Resource Description Framework (RDF) database on distributed memory
high-performance cluster. In this framework, SPARQL queries will be compiled by
SPARQL-to-C++ compiler. With a SGLib, the queries will be eventually converted
to graph pattern matching operations. This seems to be a good application of
Graph processing on RDF database, but intensive background knowledge needs to be
investigated to get better understanding of this work. 

\section{Distributed Graph Engine for Web Scale RDF Data}
In this work, Trinity.RDF \cite{zeng2013distributed} is developed for a
distributed system. Previous work usually relies on join operations for processing
SPARQL queries and a large amount of useless data will be generated. To avoid
this problem, this work models and stores the RDF data with native graph data.
The SPARQL query is represented as a query graph and the SPARQL query processing
problem is transformed to be subgraph matching. Trinity.RDF is part of the Trinity
project and more details can be found \cite{microsoft2013trinity}.
